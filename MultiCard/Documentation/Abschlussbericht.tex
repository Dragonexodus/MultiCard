%allgemeine Formatangaben
\documentclass[
 a4paper, 										% Papierformat
 12pt,												% Schriftgröße
 ngerman, 										% für Umlaute, Silbentrennung etc.
 titlepage,										% es wird eine Titelseite verwendet
 oneside, 										% einseitiges Dokument
 captions=nooneline,					% einzeilige Gleitobjekttitel ohne Sonderbehandlung wie mehrzeilige Gleitobjekttitel behandeln
 numbers=noenddot,						% Überschriften-??Nummerierung ohne Punkt am Ende
 parskip=half,									% zwischen Absätzen wird eine halbe Zeile eingefügt
 ]{scrartcl}

% Anpassung an Landessprache
\usepackage[ngerman]{babel}	

\usepackage[T1]{fontenc}	
\usepackage[utf8]{inputenc}	
\usepackage{textcomp} 																% Euro-Zeichen und andere
\usepackage[babel,german=quotes]{csquotes}						% Anführungszeichen
\RequirePackage[ngerman=ngerman-x-latest]{hyphsubst} 	% erweiterte Silbentrennung

% Befehle aus AMSTeX für mathematische Symbole z.B. \boldsymbol \mathbb
\usepackage{amsmath,amsfonts}

% Zeilenabstände und Seitenränder 
\usepackage{setspace}
\usepackage{geometry}

% Einbinden von JPG-Grafiken
\usepackage{graphicx}

% zum Umfließen von Bildern
% Verwendung unter http://de.wikibooks.org/wiki/LaTeX-Kompendium:_Baukastensystem#textumflossene_Bilder
\usepackage{floatflt}

% Verwendung von vordefinierten Farbnamen zur Colorierung
% Palette und Verwendung unter http://kitt.cl.uzh.ch/kitt/CLinZ.CH/src/Kurse/archiv/LaTeX-Kurs-Farben.pdf
\usepackage[usenames,dvipsnames]{color} 

% Tabellen
\usepackage{array}
\usepackage{longtable}

% einfache Grafiken im Code
% Einführung unter http://www.math.uni-rostock.de/~dittmer/bsp/pstricks-bsp.pdf
\usepackage{pstricks}

% Quellcodeansichten
\usepackage{verbatim}
\usepackage{moreverb} 											% für erweiterte Optionen der verbatim Umgebung
% Befehle und Beispiele unter http://www.ctex.org/documents/packages/verbatim/moreverb.pdf
\usepackage{listings} 											% für angepasste Quellcodeansichten siehe
% Kurzeinführung unter http://blog.robert-kummer.de/2006/04/latex-quellcode-listing.html

% verlinktes und Farblich angepasstes Inhaltsverzeichnis
\usepackage[pdftex,
colorlinks=true,
linkcolor=InterneLinkfarbe,
urlcolor=ExterneLinkfarbe]{hyperref}
\usepackage[all]{hypcap}

% URL verlinken, lange URLs umbrechen
\usepackage{url}

% sorgt dafür, dass Leerzeichen hinter parameterlosen Makros nicht als Makroendezeichen interpretiert werden
\usepackage{xspace}

% Beschriftungen für Abbildungen und Tabellen
\usepackage{caption}

% Entwicklerwarnmeldungen entfernen
\usepackage{scrhack}

\newcommand{\qq}[1]{\glqq{#1\grqq{}}} %Gänsefüßchen

\onehalfspacing 							% 1,5facher Zeilenabstand

\definecolor{InterneLinkfarbe}{rgb}{0.1,0.1,0.3} 	% Farbliche Absetzung von externen Links
\definecolor{ExterneLinkfarbe}{rgb}{0.1,0.1,0.7}	% Farbliche Absetzung von internen Links

% Einstellungen für Fußnoten:
\captionsetup{font=footnotesize,labelfont=sc,singlelinecheck=true,margin={5mm,5mm}}
%TODO						
\title{Abschlussbericht für das Modul SmartCard-Programmeriung}
\subtitle{Implementierung einer Mulitcard-Anwendung\vspace{1cm}}

%TODO
\author{}
\date{\today}
\begin{document}
\maketitle

\tableofcontents
\pagebreak

\section{Einleitung}
In dem Projekt \qq{Implementierung einer MultiCard} für das Wahlpflichtmodul \qq{SmartCard-Programmierung} im Sommersemester 2016 geht es um eine beispielhafte Realisierung einer SmartCard-Anwendung.
Zu einer SmartCard-Anwendung gehören sogenannte On- und OffCard-Teile.
Der OnCard-Teil beschreibt die Anwendung auf der SmartCard, während der OffCard-Teil die Nutzerseite beschreibt.

In dem Projekt \qq{Implementierung einer MultiCard} sollen mögliche Anwendungen für eine Studenten- und Discokarte realisiert werden.
Gleichermaßen soll jegliche Kommunikation kryptographisch verschlüsselt erfolgen, damit eine Manipulation der Karte beziehungsweise der Daten auf der Karte verhindert werden.

\paragraph{Aufgabe:}
Es sollte eine SmartCard-Anwendung entstehen, welche als Bestandteile zum Einen eine Studenten-Karte realisiert.
Zum Anderen soll die SmartCard eine sogenannte Disco-Karte realisieren.
Weiterhin soll die komplette Kommunikation verschlüsselt erfolgen.

Bestandteile der Student-Anwendung sind:
\begin{itemize}
	\item Studentenname und Matrikel
	\item Raumzugänge
	\item Geld einzahlen und per Essen verbrauchen
	\item Daten abfragen
\end{itemize}

Bestandteile der Disco-Anwendung sind:
\begin{itemize}
	\item Geld einzahlen
	\item Konsumierte Getränke speichern und bezahlen
	\item Bonuspunkte für Konsum und verbrauchen der Punkte
	\item Daten abfragen
\end{itemize}

Bestandteile der Kryptographie sind:
\begin{itemize}
	\item Verschlüsseln
	\item Entschlüsseln
	\item Signierung
\end{itemize}

\paragraph{Verlauf:}
Im Verlauf des Projektes wurden die Anforderungen entsprechend der Aufgabe realisiert.
Das gesamte Projekt wurde per sogenanntem \qq{Pair-Programming} realisiert.
\qq{Pair-Programming} bezeichnet das gemeinsame Programmieren an einem Bildschirm, während einer das Programm schreibt denkt der jeweils andere über das geschriebene nach.
Regelmäßig werden diese beiden Rollen getauscht.

Aufgrund dass unter Linux programmiert wurde, kann die realisierte Anwendung nicht real getestet werden, da benötigte Treiber nicht existieren.
Daher wurde das gesamte Projekt in der Simulation getestet.

\paragraph{Ergebnis:}
Als Ergebnis des Projektes entstand eine kryptographisch gesicherte SmartCard-Anwendung, welche ein Studenten-, Disco- und Crypto-Applet enthält.
Außerdem entstand eine OffCard-Anwendung welche diese Funktionalitäten der SmartCard anspricht.
Der einfachheitshalber wurde die OffCard-Anwendung in Tabs untergliedern um die verschiedenen Einsatzbereiche abzutrennen.

\paragraph{Aufbau:}
In diesem Bericht wird zunächst auf die verschiedenen Applets mit Ihren Anweisungen und den verschiedenen Fehlern eingegangen.
Im Anschluss an den OnCard-Teil wird auf den OffCard-Teil eingegangen.
Am Ende des Berichtes wird ein Fazit über das realisierte Projekte gezogen.

\section{Oncard}
Innerhalb der Applets auf der SmartCard wurden zum Teil eigene Fehler definiert.
Dies erleichtert die Lokalisierung von Fehlern und Problemen bei der Abarbeitung von Anweisungen auf der SmartCard.
Falls ein Fehler auftritt, wird dieser an die OffCard-Anwendung weitergeleitet und entsprechend behandelt.
Die selbst definierten Fehler-Codes werden nachfolgend bei den entsprechenden Applets angegeben.

Zu den selbst definierten Fehler-Codes können noch folgende allgemeine Fehler auftreten:
\begin{itemize}
	\item ISO7816.SW\_CLA\_NOT\_SUPPORTED
	\begin{itemize}
		\item Klassenbyte wird nicht unterstützt.
	\end{itemize}
	\item ISO7816.SW\_INS\_NOT\_SUPPORTED
	\begin{itemize}
		\item Anweisung wird nicht unterstützt.
	\end{itemize}
	\item ISO7816.SW\_WRONG\_LENGTH
	\begin{itemize}
		\item Daten haben falsche Länge.
	\end{itemize}
	\item ISO7816.SW\_DATA\_INVALID)
	\begin{itemize}
		\item Daten sind ungültig (wurden manipuliert).
	\end{itemize}
\end{itemize}

Der OnCard-Teil enthält die drei Applets: Student, Disco und Crypto.

\subsection{Student-Applet}

Die AID des Student-Applets ist Student in hexadezimal geschrieben:
\begin{itemize}
	\item 0x 53 74 75 64 65 6e 74
\end{itemize}
Das Klassenbyte für das Student-Applet ist:
\begin{itemize}
	\item 0x 20
\end{itemize}

Das Student-Applet unterteilt sich in drei Komponenten.
Zum Einen die Funktionalität mit Geld umzugehen und zum Anderen das Speichern von Studentennamen sowie die Raumfreischaltungen.
Das Essen in der Mensa auf der Studentenkarte wird durch einfaches Geld ausgeben realisiert.

\subsubsection{Geld}
\label{label:geld}
Im Student-Applet kann Geld eingezahlt und verbraucht werden, dabei wird Euro und Cent jeweils als 1 Byte dargestellt.
Auf der Karte wird zwischen ganzzahligem Euro-Betrag und Cent-Betrag unterschieden.
Daraus ergibt sich für den Euro-Betrag eine Obergrenze von 127\texteuro{}, da sonst ein Überlauf des Bytes erfolgen würde.
Die Höchstgrenze für den Cent-Betrag ist auf 99\textcent{} gesetzt.
Sodass der Höchstbetrag auf der Karte 127,99\texteuro{} ist.

\paragraph{INS\_GET\_MONEY (0x20)}
Liefert das aktuell auf der Karte gespeicherte Guthaben.
\paragraph{INS\_ADD\_MONEY (0x21)}
Fügt der Karte Geld hinzu.

Folgende Fehler können geworfen werden:
\begin{itemize}
	\item ERROR\_ADD\_EURO\_OVERFLOW (0xE021)
	\begin{itemize}
		\item Es kann kein ganzzahliger Betrag mehr hinzugefügt werden, da sonst ein Überlauf entstehen würde.
	\end{itemize}
	\item ERROR\_ADD\_CENT\_OVERFLOW (0xE121)
	\begin{itemize}
		\item Bei dem hinzufügen von Cents, entsteht ein Überlauf im Euro. 
	\end{itemize}
	\item ERROR\_ADD\_MONEY\_OVERFLOW (0xE221)
	\begin{itemize}
		\item Hinzufügen von Euro und Cent sorgt für einen Überlauf.
	\end{itemize}
\end{itemize}

\paragraph{INS\_SUB\_MONEY (0x22)}
Es wird ein Betrag vom Geld subtrahiert und damit ausgegeben.

Folgende Fehler können geworfen werden:
 \begin{itemize}
 	\item ERROR\_SUB\_EURO\_UNDERFLOW (0xE022)
 	\begin{itemize}
 		\item Subtraktion von Euro führt zu einem Unterlauf des Guthabens.
 	\end{itemize}
 	 \item ERROR\_SUB\_CENT\_UNDERFLOW (0xE122)
 	\begin{itemize}
 		\item Subtraktion von Cents führt zu einem Unterlauf des Guthabens.
 	\end{itemize}
 	 \item ERROR\_SUB\_INSUFFICIENT\_MONEY (0xE222)
 	\begin{itemize}
 		\item Nicht genügend Guthaben auf der Karte.
 	\end{itemize}
 \end{itemize}
 
\paragraph{INS\_RESET\_MONEY (0x23)}
Setzt Euro- und Cent-Betrag auf der Karte auf Null (Werkszustand).

\subsubsection{Studentendaten}
Auf der Karte wird der Name sowie die Matrikelnummer des Studenten gespeichert.

\paragraph{INS\_SET\_NAME (0x24)}
Schreibt den Namen des Studenten auf die Karte.
Der Name des Studenten darf maximal 50 Zeichen lang sein.

\paragraph{INS\_GET\_NAME (0x25)}
Liefert den gepeicherten Namen.

\paragraph{INS\_SET\_MATRIKEL (0x26)}
Setzt die Matrikelnummer des Studenten als Zahl (2 Byte).
Die Matrikelnummer darf höchstens 32767.

Folgende Fehler können geworfen werden:
\begin{itemize}
	\item ERROR\_SET\_MATRIKEL\_NEGATIVE (0xE025)
	\begin{itemize}
		\item Matrikelnummer darf nicht negativ sein.
	\end{itemize}
	\item ERROR\_SET\_MATRIKEL\_OVERFLOW (0xE125)
	\begin{itemize}
		\item Matrikel ist größer als höchstmöglicher Wert, wodurch ein Überlauf entstehen kann.
	\end{itemize}
\end{itemize}

\paragraph{INS\_GET\_MATRIKEL (0x27)}
Liefert die gespeicherte Matrikelnummer des Studenten.

\subsubsection{Raumfreischaltung}
Räume für den Studenten werden freigeschaltet in dem geprüft wird ob der Raum auf der SmartCard gespeichert ist (in Anlehnung an das System der HTWK Leipzig).
Damit Räume freigeschaltet werden können, werden diese als Liste auf der Karte gespeichert.
Es wird dabei immer die komplette Liste gesetzt beziehungsweise abgefragt.
Auf der Karte können maximal 16 Räume gespeichert werden.
Jeder Raum besteht aus drei Bytes (Buchstaben + Nummer).

\paragraph{INS\_SET\_ROOMS (0x28)}
Speichert die als Liste übergebenen Räume auf der Karte.
\paragraph{INS\_GET\_ROOMS (0x29)}
Liefert eine Liste von gespeicherten Räumen auf der Karte

\subsection{Disco-Applet}
Die AID des Disco-Applets ist Disco in hexadezimal geschrieben:
\begin{itemize}
	\item 0x 44 69 73 63 6f
\end{itemize}
Das Klassenbyte für das Disco-Applet ist:
\begin{itemize}
	\item 0x 30
\end{itemize}

Das Disco-Applet besteht im Wesentlichen aus drei Komponenten.
Zum Einen das Sparen von Bonuspunkten durch Konsum von Getränken sowie das Speichern der verbrauchten Getränke auf der Karte.
Zum Anderen müssen die Getränke mit Hilfe von Geld (aus Student-Applet) und Bonuspunkten bezahlt werden. 
Üblicherweise wird die SmartCard bei Betreten der Lokalität vorgelegt und der Eintritt bezahlt.
Beim Verlassen der Lokalität wird die Karte wieder vorgelegt und entsprechender Getränkekonsum über die Karte bezahlt.

\subsubsection{Bonus}
Der Bonus wird als ein Byte gespeichert, daraus ergibt sich der maximale Bonus zu 127, da sonst ein Überlauf auftreten kann.
Bonuspunkte werden durch das konsumieren erhöht.
Die OffCard-Anwendung kümmert sich um die Verwaltung der Bonuspunkte.

\paragraph{INS\_GET\_BONUS (0x30)}
Liefert die aktuell gespeicherten Bonuspunkte.
\paragraph{INS\_ADD\_BONUS (0x31)}
Fügt Bonuspunkte hinzu.

Folgende Fehler können geworfen werden:
\begin{itemize}
	\item ERROR\_ADD\_BONUS\_OVERFLOW (0xE030)
	\begin{itemize}
		\item Es kann kein Bonus mehr hinzugefügt, da sonst ein Überlauf auftreten würde.
	\end{itemize}
\end{itemize}
\paragraph{INS\_SUB\_BONUS (0x32)}
Verbraucht Bonuspunkte.

Folgende Fehler können geworfen werden:
\begin{itemize}
	\item ERROR\_SUB\_BONUS\_UNDERFLOW (0xE031)
	\begin{itemize}
		\item Subtrahieren von Bonus würde ein Unterlauf-Fehler verursachen.
	\end{itemize}
	\item ERROR\_SUB\_INSUFFICIENT\_BONUS (0xE032)
	\begin{itemize}
		\item Nicht genug Bonuspunkte auf der Karte.
	\end{itemize}
\end{itemize}
\paragraph{INS\_RESET\_BONUS (0x33)}
Setzt die Bonuspunkte auf der Karte auf Null (Werkszustand).

\subsubsection{Geld}
Auf dem Disco-Applet können Geldfunktionen aus dem Student-Applet genutzt werden.
Dies ist nötig damit Student und Disco nicht über getrennte Guthaben verfügen müssen.
Kommunikation und der Zugriff auf das Geld des Student-Applet erfolgt über die Applet-Firewall.
Die Anweisungen sind identisch zu den Anweisungen auf dem Student-Applet.
Für Details zu den Anweisungen wird auf \autoref{label:geld} verwiesen.

\subsubsection{Getränke}
Auf der SmartCard können maximal 50 Getränke gespeichert werden, welche in der Disco konsumiert wurden.
Für ein Getränk wird nur die ID als Byte in einer Liste gespeichert.
Der Preis zu dem Getränk wird durch die OffCard-Anwendung der ID zugeordnet.
	
\paragraph{INS\_GET\_DRINKS (0x34)}
Liefert eine Liste von konsumierten Getränken.
\paragraph{INS\_ADD\_DRINK (0x35)}
Fügt ein Getränk der Liste hinzu.

Folgender Fehler kann geworfen werden:
\begin{itemize}
	\item ERROR\_ADD\_DRINK\_HAD\_TO\_MUCH (0xE033)
	\begin{itemize}
		\item Getränkeliste ist voll, es können keine weiteren Getränke hinzugefügt werden.
	\end{itemize}
\end{itemize}

\paragraph{INS\_SET\_PAID\_DRINKS (0x36)}
Anhand einer übergebenen Liste, welche Einsen oder Nullen enthält, wird die interne Getränkeliste bereinigt, um Getränke welche bezahlt wurden.
Eine Eins entspricht dabei bereits bezahlt und eine Null noch zu bezahlen.
Die Übergebene Liste muss die gleiche Länge haben wie die derzeit gespeicherte Liste auf der Karte.

\subsection{Crypto-Applet}
%TODO
%private static final short ERROR_IMPORT_CARD_PRIVATE_MOD = (short) 0xE010;
%private static final short ERROR_IMPORT_CARD_PRIVATE_EXP = (short) 0xE011;
%private static final short ERROR_IMPORT_CARD_PUBLIC_MOD = (short) 0xE012;
%private static final short ERROR_IMPORT_CARD_PUBLIC_EXP = (short) 0xE013;
%private static final short ERROR_EXPORT_CARD_PUBLIC_MOD = (short) 0xE014;
%private static final short ERROR_EXPORT_CARD_PUBLIC_EXP = (short) 0xE015;
%private static final short ERROR_IMPORT_TERMINAL_PUBLIC_MOD = (short) 0xE016;
%private static final short ERROR_IMPORT_TERMINAL_PUBLIC_EXP = (short) 0xE017;


Die AID des Crypto-Applets ist Crypto in hexadezimal geschrieben:
\begin{itemize}
	\item 0x 43 72 79 70 74 6f
\end{itemize}
Das Klassenbyte für das Crypto-Applet ist:
\begin{itemize}
	\item 0x 10
\end{itemize}

\paragraph{INS\_IMPORT\_CARD\_PRIVATE\_MOD (0x10)}
\paragraph{INS\_IMPORT\_CARD\_PRIVATE\_EXP (0x11)}
\paragraph{INS\_IMPORT\_CARD\_PUBLIC\_MOD (0x12)}
\paragraph{INS\_IMPORT\_CARD\_PUBLIC\_EXP (0x13)}
\paragraph{INS\_EXPORT\_CARD\_PUBLIC\_MOD (0x14)}
\paragraph{INS\_EXPORT\_CARD\_PUBLIC\_EXP (0x15)}
\paragraph{INS\_IMPORT\_TERMINAL\_PUBLIC\_MOD (0x16)}
\paragraph{INS\_IMPORT\_TERMINAL\_PUBLIC\_EXP (0x17)}


%TODO
Der komplette Datenaustausch zwischen SmartCard und OffCard-Anwendung soll verschlüsselt und signiert geschehen. 
Damit nicht jedes Applet die dafür notwendige Logik implementieren muss, wurde mit dem Crypto-Applet eine zentrale Anlaufstelle für folgende Aufgaben geschaffen:

verschlüsseln und signieren

entschlüsseln und verifizieren

Als Kryptosystem wird RSA mit einer Schlüssellänge von 512 Bit eingesetzt. Ursprünglich war eine Schlüssellänge von 1024 Bit angedacht, jedoch resultierte daraus ein Ciphertext von 128 Byte. Zusammen mit der Signatur entstehen somit 256 Byte an zu versendenden Daten. Da die vorliegenden Smart-
Cards jedoch nur 255 Byte an Daten unterstützen, wurde sich für eine Reduzierung der Schlüssellänge entschieden.
Das Cryptography-Applet stellt folgende öffentlich zugängliche Anweisungen bereit:


Wie an den Anweisungsnamen erkennbar ist, ist es möglich, das Schlüsselpaar bestehend aus privaten und öffentlichen Schlüssel für die Karte zu importieren. Dies ist notwendig, da sonst die Signierung nicht als Sicher eingestuft werden kann.
Nach der Installation der Applets befindet sich die Karte in ihrem Werkszustand. Es sind keine Schlüsselpaare und auch keine Daten auf der Karte gesetzt. 
Um die Karte benutzen zu können, müssen nun
als erstes die Schlüssel für das RSA Kryptosystem gesetzt werden. 
Dazu wird die OffCard-Anwendung genutzt. 
Eine nachträgliche Änderung der Schlüssel wird mit Hilfe von Flags unterbunden. Die Karte ist somit gebrandmarkt.
Im gesamten Hotel existiert ein Schlüsselpaar für die Karten und ein Schlüsselpaar für die Terminals.
Auch wenn eine dritte Partei eine Karte im Werkszustand in die Hand bekommen und seine eigenen Schlüssel setzen sollte bleibt das System sicher. Es ist nicht möglich, an die Karte gesendete Daten zu entschlüsseln, da der private Schlüssel der Karte falsch ist sowie die Signatur nicht mit dem privaten
Schlüssel der Terminals verifiziert werden kann. 
Aufgrund der nicht passenden Schlüssel ist es ebenso wenig möglich, gefälschte Daten an die OffCard-Anwendung zu schicken. Das System wird erst unsicher, wenn die Schlüsselpaare für Karten und Terminals bekannt würden. Mit ihnen ist es dann möglich vertrauenswürdige Karten zu fälschen.
Die Methoden für die Ver- und Entschlüsselung sind innerhalb der Karte über die Applet-Firewall zugänglich. Den Applets Student und Disco ist es erlaubt, eine Instanz des Crypto-Applets zu erhalten. 
Je nach Richtung der Datenübertragung können diese Applets dann entweder Daten ver- oder entschlüsseln. 
%Bei ein- und ausgehenden verschlüsselten Daten hat die APDU folgenden Aufbau:
%CLA
%INS
%P1
%P2
%LC (0x80)
%64 Byte verschlüsselte Daten
%64 Byte Signatur
%LE

Beim Aufruf der Entschlüsselungs-Methode (decrypt) werden die 64 Bit Daten mit dem privaten Schlüssel der Karte entschlüsselt. 
Der dadurch gewonnene Klartext wird mithilfe des öffentlichen Schlüssels des Terminals und der mitgesendeten Signatur verifiziert. 
Der Klartext wird für die weitere Verwendung im Puffer abgelegt. 
%Schlägt die Verifizierung fehl, wird die ISO-ExceptionISO7816.SW_DATA_INVALID ausgelöst. 
In diesem Fall ist davon auszugehen, dass die Daten manipuliert wurden.
Beim Aufruf der Verschlüsslungs-Methode (encrypt) werden die in die Methode übergebenen Daten mit dem privaten Schlüssel der Karte signiert. 
Weiterhin werden die Daten mit dem öffentlichen Schlüssel des Terminals verschlüsselt. 
Daten und Signatur werden im Puffer abgelegt und können vom aufrufenden Applet versendet werden.
Im Gegensatz zu allen anderen Applets ist für das Crypto-Applet keine Reset Möglichkeit vorgesehen. Um die Schlüssel neu setzen zu können, muss das Applet neu installiert werden.

\section{OffCard}
%TODO MINDESTAFORDERUNGEN Java 8,60
%TODO AUFBAU,Bilder, ANWENDUNG, FUNKTIONSWEISE
\subsection{Simulation}
%TODO
Für die Benutzung der OffCard-Anwendung mit der simulierten SmartCard muss in der opencard.properties Datei die Konfiguration für die Simulation aktiv sein.
Die Konfiguration für die reale SmartCard muss mit Zeilenkommentaren (\#) deaktiviert werden.
Die Simulation der SmartCard muss im Eclipse gestartet werden.
Anschließend muss das Terminal mit dem Port 8050 geöffnet werden. 
Dazu kann folgender Befehl verwendet werden:
/terminal "Remote|localhost:8050"
Nach dem Freigeben der Verbindung mit /close kann die OffCard-Anwendung gestartet und benutzt werden.

\subsection{Connection-Tab}
todo
%On-Voraussetzung: alle Applets instalieren und /close ausführen und vor (nur Simulation)
\subsection{Configuration-Tab}
todo
\subsection{Student-Tab}
todo
\subsection{Disco-Tab}

\section{Fazit}
todo
\end{document}